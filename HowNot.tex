\documentclass[runningheads]{llncs}

% 5 + 4 + 2 = non-exp + exp + refs
\begin{document}
\title{How NOT To Evaluate Your Dialogue System: An Empirical Study of
Unsupervised Evaluation Metrics for Dialogue Response Generation}
\titlerunning{How NOT To Evaluate Your Dialogue System}

\author{Chia-Wei Liu \inst{1} \and Ryan Lowe \inst{1} \and
Iulian V. Serban \inst{2} \and Micheal Noseworthy \inst{1}}

\institute{School of Computer Science, McGill University
\email{\{chia-wei.liu,ryan.lowe,michael.noseworthy\}@mail.mcgill.ca}
\and
DIRO, Universite de Montreal \\
\email{iulian.vlad.serban@umontreal.ca}}

\maketitle

\begin{abstract}
    We investigate evaluation metrics for dialogue
    response generation systems where supervised
    labels, such as task completion, are not available.
    Recent works in response generation
    have adopted metrics from machine translation
    to compare a model’s generated response
    to a single target response. We show that
    these metrics correlate very weakly with human
    judgements in the non-technical Twitter
    domain, and not at all in the technical Ubuntu
    domain. We provide quantitative and qualitative
    results highlighting specific weaknesses
    in existing metrics, and provide recommendations
    for future development of better automatic
    evaluation metrics for dialogue systems.
\end{abstract}

\section{Introduction}
\paragraph{1.} Two kinds of dialogue response generation systems:
supervised dialogue models and unsupervised dialogue models.

\paragraph{2. } The focus of this paper,
the unsupervised models such as chatbots using end-to-end training
with neural networks can avoid supervised labels but their
evaluation is an open problem.

\paragraph{3. } Other NLP tasks have developed effective automatic metrics such as:
\begin{enumerate}
    \item Machine translation: BLEU and METEOR.
    \item Automatic Summarization: ROUGE.
\end{enumerate}
But the assumption these metrics make is too strong for dialogue systems, which
permit vastly diverse responses to a given context.

\paragraph{4. } The main work of this paper: investigate the correlation
between several metrics and human judgement on two datasets and four models.
The \textbf{experiment objects} of this paper is as follow:
\begin{enumerate}
    \item Models:
    \begin{enumerate}
          \item Retrieval: TF-IDF, Dual Encoder.
        \item Generative: LSTM, HRED.
    \end{enumerate}
    \item Metrics:
    \begin{enumerate}
        \item Word-Overlap: BLEU, METEOR, ROUGE.
        \item Word-Embedding: ...
    \end{enumerate}
    \item Datasets:
    \begin{enumerate}
        \item Chitchat-Oriented: Twitter dataset.
        \item Technical-Oriented: Ubuntu Dialogue Dataset.
    \end{enumerate}
\end{enumerate}

Major conclusion:
For all metrics, there is weak correlation on Twitter dataset and no correlation
on Ubuntu dataset.
Methodology:
\begin{enumerate}
    \item statistical analysis on the survey's results.
    \item qualitative analysis of examples.
    \item exploration of the sensitivity of the metrics.
\end{enumerate}

\paragraph{5. } Conclusion of the experiment:
\begin{enumerate}
    \item The advice to the researchers: shift away from these metrics.
    \item We need to find a new metric that correlates more strongly with human.
\end{enumerate}

\section{Related Work}
% Find good reasons why some of the metrics are not used.
\paragraph{1. } Excluding certain metrics and the reasons of doing so.
\begin{enumerate}
    \item Perplexity
    \begin{enumerate}
        \item Violate \emph{model-independent}. The model generating the response
        does not evaluate itself.
        \item There is no utterance level score for PPL.
        \item Cannot be computed for retrieval models.
    \end{enumerate}
    \item Recall
    \begin{enumerate}
        \item The pattern of metrics should be evaluating the response against the
        ground-truth (The setting in machine translation).
    \end{enumerate}
    \item Metrics for supervised models.
    \begin{enumerate}
        \item We don't study supervised metrics in this paper.
        \item These metrics are well-studied.
    \end{enumerate}
\end{enumerate}

\paragraph{2.} Usage of metrics in the lierature.
\begin{enumerate}
    \item Ritter et al. (2011) use an SMT model to beat retrieval baselines
    under BLEU.
    \item Sordoni et al. (2015) use an RNNLM model and context-sensitive multi-BLEU.
    \item Li et al. (2015) evaluate their diversity-promoting objective function
    with single-reference BLEU.
    \item Galley et al. (2015b) propose deltaBLEU that uses multiple human evaluated
    ground-truth. But it only has weak to moderate correlation to human judgement.
\end{enumerate}

\paragraph{3.} There are studies
of correlation of metrics with human judgements in other NLP fields,
such as MT and NLG. But dialogue response generation is intrinsically
a harder problem since the set of correct answers are large.
Studies are listed as follow:
\begin{enumerate}
    \item MT: (Callison-Burch et al., 2010; Callison-Burch et al.,
2011; Bojar et al., 2014; Graham et al., 2015).
    \item NLG: (Stent et
al., 2005; Cahill, 2009; Reiter and Belz, 2009; Es-
pinosa et al., 2010).
\end{enumerate}

\section{Evaluation Metrics}
\subsection{Word Overlap-based Metrics}
\paragraph{BLEU.}
\paragraph{METEOR.}
\paragraph{ROUGE.}

\subsection{Embedding-based Metrics}
\paragraph{Greedy Matching.}
\paragraph{Embedding Average.}
\paragraph{Vector Extrema.}


\section{Dialogue Response Generation Models}
\subsection{Retrieval Models}
\paragraph{TF-IDF}
\paragraph{Dual Encoder}

\subsection{Generative Models}
\paragraph{LSTM language model.}
\paragraph{HRED}
\subsection{Conclusions from an Incomplte Analysis}

\section{Human Correlation Analysis}
\paragraph{Data Collection.}
\paragraph{Survey Results.}
\paragraph{Qualitative Analysis.}

\section{Discussion}
\paragraph{Constrained tasks.}
\paragraph{Incoporating multiple responses.}
\paragraph{Searching for suitable metrics.}

\end{document}
